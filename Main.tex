\documentclass[12pt,a4paper]{scrartcl}
\usepackage{geometry}                % See geometry.pdf to learn the layout options. There are lots.
\geometry{a4paper}                   % ... or a4paper or a5paper or ... 
%\geometry{landscape}                % Activate for for rotated page geometry
%\usepackage[parfill]{parskip}    % A-bombctivate to begin paragraphs with an empty line rather than an indent
\usepackage{graphicx}
\usepackage{amssymb}
\usepackage{epstopdf}
%Eigene Packages
\usepackage{ngerman}			% Deutsch als Sprache (Silbentrennung etc)
\usepackage{longtable}			% Unterstützung für Seitenumfassende Tabellen
\usepackage{gensymb}			% Symbole
\usepackage{ae}				% Besseres Schriftbild in PDF Dateien
\usepackage[pdftex]{hyperref}		% Paket für klickbares Inhaltsverzeichnis, Miniaturen usw
\usepackage[utf8]{inputenc}	% Paket für Umlaute (utf8 encoding), alte haben unter Umständen noch applemac! oder noch schlimmer: westeuropäisch windows
\usepackage [T1]{fontenc}		
% eigene Packages
% Pakete von Rouven
\usepackage{listings}
\usepackage{color}
\usepackage{amsmath}

%color Definitionen
\definecolor{middlegray}{rgb}{0.5,0.5,0.5}
\definecolor{lightgray}{rgb}{0.8,0.8,0.8}
\definecolor{orange}{rgb}{0.8,0.3,0.3}
\definecolor{yac}{rgb}{0.6,0.6,0.1}
 
% Listings Settings
 \lstset{
   language=c++,
   backgroundcolor=\color{lightgray},
   basicstyle=\scriptsize\ttfamily,
   keywordstyle=\bfseries\ttfamily\color{orange},
   stringstyle=\color{blue}\ttfamily,
   commentstyle=\color{middlegray}\ttfamily,
   emph={square}, 
   emphstyle=\color{blue}\texttt,
   emph={[2]root,base},
   emphstyle={[2]\color{yac}\texttt},
   showstringspaces=false,
   flexiblecolumns=false,
   tabsize=2,
   numbers=none,
   numberstyle=\tiny,
   numberblanklines=false,
   stepnumber=1,
   numbersep=10pt,
   xleftmargin=15pt
 }
 
 \lstset{literate=%
{Ö}{{\"O}}1 
{Ä}{{\"A}}1 
{Ü}{{\"U}}1 
{ß}{{\ss}}2 
{ü}{{\"u}}1 
{ä}{{\"a}}1 
{ö}{{\"o}}1
}
 
 \hypersetup
{%
pdftitle = {Algorithmen und Programme},
pdfsubject = {AuP},
pdfauthor = {Rouven Czerwinski},
pdfkeywords = {AuP, Algorithmen, Programme},
colorlinks = {true},
linkcolor={blue},
anchorcolor={black}
}
% Eigene Befehle
\newcommand\txtunderbrace[2]{%
  \settowidth{\myx}{#2}%
  \begin{tabular}[t]{@{}>{\centering}p{\myx}@{}}%
    \ensuremath{\underbrace{\text{#2}}}\tabularnewline
    \makebox[0pt][c]{#1}~\tabularnewline
  \end{tabular}%
}%\txtunderbrace{Text unter der Klammer}{Text im Kontext}

\DeclareGraphicsRule{.tif}{png}{.png}{`convert #1 `dirname #1`/`basename #1 .tif`.png}

\title{Algorithmen und Programme}
\author{Protokolliert von Rouven Czerwinski}
\date{Version vom \today}                                           % Activate to display a given date or no date

\begin{document}
\maketitle
\newpage

\tableofcontents
\listoftables
\newpage

\section{Einführung}

\begin{itemize}
 \item Kleinstcomputer (eingebettete Systeme) mit Alg. in allen Bereichen des täglichen Lebens: Taschenrechner, Handy, DvD-Player, MP3-Player, Waschmaschine, TV, Autos, Funkuhren\dots
 \item Programmierkenntnisse werden erwartet:
  \begin{itemize}
   \item[-] Programmierung und Steuerung komplexer Geräte und Maschinen
   \item[-] Erstellung interaktiver Medien (Internet, Videospiele, DVD, BluRay, E-Books\dots )
   \item[-] Verwaltung und Auswertung von Datenbanken
  \end{itemize}
\end{itemize}

\subsection{Algorithmusbegriff}
Intuitiv: Alg. = Verarbeitungsvorschrift \\
Im Alltag z.B. Kochrezept, Spielregeln, Noten, Waschmaschinenprogramme, \dots \\
Man spricht von einem Alg., wenn die Vorschrift \underline{präzise}, \underline{eindeutig}, \underline{vollständig} und \underline{ausführbar} ist. \\
~\\
\underline{Definition:}

\begin{quote}
Ein Alg. ist eine präzise formulierte Verarbeitungsvorschrift, die unter Verwendung elementarer Operationen einen Eingangszustand bzw. Einganswerte in einen Ausgangszustand bzw. Ausgangswerte überführt
\end{quote}
~\\
\underline{Formal:} Abbildung f: Eingabe $\rightarrow$ Ausgabe \\
~\\
\underline{Beispiele:}
\begin{itemize}
 \item Mathematische Formeln: $f: \mathbb{R} \times \mathbb{R} \rightarrow \mathbb{R}$ z.B. Addition zweier Zahlen $f(q,p) = q+p$
 \item Primzahlentest: $f: \mathbb{N} \rightarrow \{ ja, nein\}$ \\
  \begin{quote}
   $ f(n)=\left\{\begin{array}{cl}  \mbox{ja, falls }n \mbox{ Primzahl} \\ \mbox{nein, sonst} \end{array}\right. $
  \end{quote}
 \item  Euklidischer Alg.  ggT(x,y)
\end{itemize}
Alg. dienen zur Lösung von Problemen, sie werden als Programme so abgefasst, dass sie von einem Rechner ausgeführt werden können: \\
\begin{quote}
  $ \underbrace{\mbox{Problem} \rightarrow \mbox{Algorithmus} \rightarrow \mbox{Progr}}\mbox{amme} \rightarrow \mbox{Maschine} $
\end{quote}
Gegenstand der Vorlesung


\newpage
\section{Algorithmische Grundkonzepte}
 \subsection{Eigenschaften von Algorithmen}
  \begin{itemize}
   \item \underline{Terminiertheit} \\
    Ein Alg. terminiert, wenn er für jede Wahl von gültigen Eingabewerten nach endlich vielen Schritten anhält
   \item \underline{Determiniertheit} \\
    Ein Alg. ist determiniert, wenn er bei gleicher Eingabe stets auf das gleiche Ergebnis führt.
   \item \underline{Determinismus} \\
    Ein Alg. ist deterministisch wenn er bei gleicher Eingabe stets über die gleichen Zwischenergebnisse zum gleichen Ergebnis führt.
    \item Beispiel: Berechnung eines Terms
    \begin{itemize}
     \item[-] hält immer an $\Rightarrow$ terminiert
     \item[-] gleiches Ergebnis $\Rightarrow$ determiniert \\
      $\Rightarrow$ nicht deterministisch
    \end{itemize}
   \end{itemize}
 \subsection{Daten, Operanden und Operationen}
 Daten:
 \begin{itemize}
 \item Darstellung von Informationen im Rechner zur Eingabe, Verarbeitung und Ausgabe
 \item zB. Zahlen, Zeichen, Texte, Tabellen, Graphen, Bilder, \dots
 \item Rechnerinterne Darstellung zB. (komprimiert vs. unkomprimiert)
 \end{itemize}
 Datentyp:
 \begin{itemize}
  \item Zusammenfassung von Wertebereich und darauf def. Operationen zu einer Einheit
  \item Beispiel: Standarddatentypen: int, float, char, \dots
  \item Ein Alg. lässt sich auffassen als Anwenden von Operationenauf Objekte bestimmten Datentyps (=Operanden).
  \item Operand können Konstanten, Variablen oder Ausdrücke sein.
  \item Ausdrücke (Terme) entstehen indem Operanden mit Operationen verknüpft werden
  \item Datentypen legen die Wertemenge fest, aus der die Operanden Werte annehmen können
 \end{itemize}
\newpage
\section{Imperative Algorithmen}
\begin{itemize}
 \item Basis für imperative Programmiersprachen wie C, Pascal, Modula, Basic, PHP, \dots
 \item bekannteste/häufigste Art Alg. zu formulieren
 \item Alternative: Deklarative Programmierung z.B. \\
 funktionale Programmiersprachen wie z.B \\
 Lisp, Scheme, Haskell, \dots
\end{itemize}

\subsection{Ein- und Ausgabe}
 C++ Beispiel
 \begin{lstlisting}
int a;
cin >> a; //Eingabe des Wertes von a ueber Konsole
cout << "Sie haben" << a << "eingegeben" << ende; // Ausgabe
 \end{lstlisting}

\subsection{Verzweigungen (bedingte Anweisung)}
Mit Verzweigungen können in Abhängigkeit einer Bedingung unterschiedliche Anweisungen ausgeführt werden. \\
\underline{Syntax:} \\
\begin{lstlisting}
if (BEDINGUNG) ANWEISUNG;
else ANWEISUNG2;
\end{lstlisting}
\underline{Interpretation:} \\
Führe ANWEISUNG1 aus, falls die boolsche BEDINGUNG wahr (true) ist, ansonsten führe ANWEISUNG2 aus. (Der Else-Teil ist optional) \\
C++ Beispiel:
\begin{lstlisting}
if (a<0) cout << "a ist negativ" << endl;
else cout << "a ist positiv" << endl;

bool b = ((a % 2) == 0); // b wird true, falls a modulo 2 null ist
if(b) cout << "a ist gerade" << endl;
\end{lstlisting}

~\\
Eine ANWEISUNG darf auch ein mit Klammern \{\} zusammengefasster Block von mehreren Anweisungen sein.
\newpage
\underline{Syntax:} \\
\begin{lstlisting}
if (BEDINGUNG)
{
	Anweisung 1.1
	Anweisung 1.2
}
\end{lstlisting}

\begin{table}[h]
	\caption[ModuloTabelle]{Modulo Tabelle}
	\begin{center}
	\begin{tabular}{c|c|c|c|c}
		a & $a/2$ & a\%2 & $a/3$ & a\%3 \\
		\hline
		0 & 0 & 0 & 0 & 0\\
		1 & 0 & 1 & 0 & 1 \\
		2 & 1 & 0 & 0 & 2 \\
	\end{tabular}
	\end{center}
\end{table}

\newpage
\section{Komplexität von Algorithmen}
\subsection{Einführung}
Alg1:
\begin{lstlisting}
void Alg1(int n)
{
	for(int i=0; i < n*4; i=i+1)
	{
		tue_etwas(...);
	}
}
\end{lstlisting}
Alg2:
\begin{lstlisting}
void Alg2(int n)
{
	for(int i=0; i < n; i=i+1)
	{
		for(int j=0; j<n; j=j+)
		{
			tue_etwas(...);
		}
	}
}
\end{lstlisting}
Aufwandsvergleich(Anzahl der Aufrufe von \texttt{tue\_etwas}): \\
\begin{table}[h]
	\caption[Aufwandsvergleich der Algorithmen]{Aufwandsvergleich}
	\begin{center}
	\begin{tabular}{c|ccccccc}
		n & 1 & 2 & 3 & 4 & 5 & 6 & 7\\
		\hline
		Alg1 & 4 & 8 & 12 & 16 & 20 & 24 & 28\\
		Alg2 & 1 & 4 & 9 & 16 & 25 & 36 & 49\\
	\end{tabular}
	\end{center}
\end{table}
~\\
Alg1 $\longleftarrow$ linear (4n) \\
Alg2 $\longleftarrow$ quadratisch ($n^2$) \\
\begin{itemize}
\item Aufwand abhängig von
	\begin{itemize}
	\item[-] Größe n der Eingabe, Anzahl der Daten usw.
	\item[-] der Komplexität des Algorithmus
	\end{itemize}
\item Unterscheidung zwischen 
	\begin{itemize}
	\item[(a)] Zeitkomplexität (Laufzeit der Alg. s.o.)
	\item[(b)] Speicherkomplexität (benötigte Speichermenge)
	\end{itemize}
\item 90/10-Regel beim Programmieren: \\ 
\glqq Ungefähr 90\% der Laufzeit wird in ca. 10\% des Programmcodes verbraucht\grqq
\end{itemize}

\subsection{Asymptotische Komplexität und Groß-O-Notation}
\begin{itemize}
\item Abschätzung der Komplexität als Funktion von $n$
\item Idee: \grqq Wachstumsverhalten\glqq möglichst allgemein für große n beschreiben\\
(Konstante Faktoren und Summanden werden nicht berücksichtigt) \\
(Zeichnung als Foto auf Handy)
\item Die Groß-O-Notation weist einer Funktion $g: \mathbb{N} \rightarrow \mathbb{N}$ eine Menge von Funktionen $O(g(n))$ zu, die in gleicher Wachstumsbeziehung zu $g$ stehen.
\item Eine Funktion $f$ ist Element von $O(g(n))$, falls sie nicht wesentlich schneller als $g$ wächst. \\
Für große $n$ muss gelten $f(n) \leq c-g(n)$ mit einer beliebigen Konstanten $c$.
\item Hier z.B. $g(n)$ wächst so schnell wie $f(n)$; \\
$\Rightarrow f(n) \in O(g(n))$ \\
in Bsp. aus 4.1. \\
Alg1: $f(n) = 4n \Rightarrow f(n) \in O(n)$ \\
Alg2: $f(n) = n^2 \Rightarrow f(n) \in O(n^2)$ \\ 
\item \underline{formale Definition} \\
$O(g(n))$ ist die Menge aller Funktionen $f(n)$, die asymptotisch beschränkt sind durch ein beliebiges aber konstantes Vielfaches der Funktion $g(n)$. \\
Mathematisch: \\
$O(g(n)) = \{ f(n) | \mbox{Es gibt positive Konstanten } c \mbox{ und } n_0 \mbox{,so dass für alle } n \ge n_0 \mbox{ gilt:} f(n) \le c*g(n) \}$
\begin{itemize}
\item[-] $O(a)$ Konstanter Aufwand
\item[-] $O(log(n))$ logarithmischer Aufwand
\item[-] $O(n)$ linearer Aufwand
\item[-] $O(n^2)$ quadratischer Aufwand
\item[-] $O(n^k)$ polynomialer Aufwand
\item[-] $O(2^n)$ exponentieller Aufwand
\item[-] $O(n^n)$ exponentieller Aufwand
\end{itemize}
\item Exponentielle Probleme sind im Allgemeinen nicht lösbar für große $n$.
\item Wichtig: Reduktion der Komplexität durch Entwicklung effizienter Algorithmen.\\
z.B. Fouriertransformation $O(n^2) \rightarrow$ schnelle Fouriertransformation (FFT): $O(n*log(n))$
\item Rechenregeln:
\begin{itemize}
\item[-] Linearität: $O(c_1*g(n)+c_2) = O(g(n)) \Rightarrow 4n+3 \in O(n)$
\item[-] Distributivität: $O(g_1(n)) + O(g_2(n)) = O(g_1(n) +g_2(n))$ \\
$O(g_1(n))*O(g_2(n)) = O(g_1(n)*g_2(n))$
\item[-] Außerdem: $O(O(g(n))) = O(g(n))$ \\
$g_1(n)*O(g_2(n)) = O(g_1(n)*g_2(n))$
\item[-] Merke: $f(n) = a_kn^k + a_{k-1}n^{k-1} + \dots + a_1n + a_0$ \\
$\Rightarrow f(n) \in O(n^k) \Rightarrow 3n^5 + 2n^2 - 18n +5 \in O(n^5)$
\end{itemize}
\end{itemize}

\subsection{Beispiele}
\underline{Berechnung von $x^n$ (Version 1)} \\
~\\
C/C++ Beispiel
\begin{lstlisting}
double pow1(double x, int n)
{
	double y = 1;
	for(int i=0; i <n; i = i+1) // n Durchläufe
	{
		y = y*x;
	}
	return y;
}
\end{lstlisting}
Schleife wird ($n$)-mal durchlaufen $1*x*x*x*x*x*\dots *x$ \\
$\Rightarrow$ Zeitkomplexität $f(n)=1 \Rightarrow f(n) \in O(n)$

\underline{Berechnung von $x^n$ (schneller)}
Rekursive Lösung des Problems durch fortlaufende Halbierung des Exponenten. \\
z.B. $x^8 = x^4*x^4 \rightarrow x^4=x^2*x^2 \rightarrow x^2=x*x$ \\
anstatt 8 nur noch 3 Multiplikationen \\
$x^9 = x^4*x^4*x \rightarrow x^4=\dots$ \\
C/C++
\begin{lstlisting}
double pow2(double x, int n)
{
	if(n==0) return 1; // x^0 = 1
	if(x==1) return x; // x^1 = x
	double y = pow2(x, n/2) // Rekursionsaufruf
	y = y*y;
	if(n%2=1) y=y*x; //für ungerade n
	return y;
}
\end{lstlisting}
Wie oft lässt sich n halbieren? \\
$\rightarrow$ Logarithmus n (zur Basis 2) \\
$\Rightarrow$ Zeitkomplexität $f(n) = log_2n$ \\
$\Rightarrow$ $f(n) \in O(log n)$ \\
Algorithmisches Prinzip: \underline{\glqq Divide and Conquer\grqq}
\end{document}