\documentclass[12pt,a4paper]{scrartcl}
\usepackage{geometry}                % See geometry.pdf to learn the layout options. There are lots.
\geometry{a4paper}                   % ... or a4paper or a5paper or ... 
%\geometry{landscape}                % Activate for for rotated page geometry
%\usepackage[parfill]{parskip}    % A-bombctivate to begin paragraphs with an empty line rather than an indent
\usepackage{graphicx}
\usepackage{amssymb}
\usepackage{epstopdf}
%Eigene Packages
\usepackage{ngerman}			% Deutsch als Sprache (Silbentrennung etc)
\usepackage{longtable}			% Unterstützung für Seitenumfassende Tabellen
\usepackage{gensymb}			% Symbole
\usepackage{ae}				% Besseres Schriftbild in PDF Dateien
\usepackage[pdftex]{hyperref}		% Paket für klickbares Inhaltsverzeichnis, Miniaturen usw
\usepackage[utf8]{inputenc}	% Paket für Umlaute (utf8 encoding), alte haben unter Umständen noch applemac! oder noch schlimmer: westeuropäisch windows
\usepackage [T1]{fontenc}		
% eigene Packages
% Pakete von Rouven
\usepackage{listings}
\usepackage{color}

%color Definitionen
\definecolor{middlegray}{rgb}{0.5,0.5,0.5}
\definecolor{lightgray}{rgb}{0.8,0.8,0.8}
\definecolor{orange}{rgb}{0.8,0.3,0.3}
\definecolor{yac}{rgb}{0.6,0.6,0.1}
 
% Listings Settings
 \lstset{
   language=c++,
   backgroundcolor=\color{lightgray},
   basicstyle=\scriptsize\ttfamily,
   keywordstyle=\bfseries\ttfamily\color{orange},
   stringstyle=\color{blue}\ttfamily,
   commentstyle=\color{middlegray}\ttfamily,
   emph={square}, 
   emphstyle=\color{blue}\texttt,
   emph={[2]root,base},
   emphstyle={[2]\color{yac}\texttt},
   showstringspaces=false,
   flexiblecolumns=false,
   tabsize=2,
   numbers=none,
   numberstyle=\tiny,
   numberblanklines=false,
   stepnumber=1,
   numbersep=10pt,
   xleftmargin=15pt
 }
 
 \hypersetup
{%
pdftitle = {Algorithmen und Programme},
pdfsubject = {AuP},
pdfauthor = {Rouven Czerwinski},
pdfkeywords = {AuP, Algorithmen, Programme},
colorlinks = {true},
linkcolor={blue},
anchorcolor={black}
}
% Eigene Befehle
\newcommand\txtunderbrace[2]{%
  \settowidth{\myx}{#2}%
  \begin{tabular}[t]{@{}>{\centering}p{\myx}@{}}%
    \ensuremath{\underbrace{\text{#2}}}\tabularnewline
    \makebox[0pt][c]{#1}~\tabularnewline
  \end{tabular}%
}%\txtunderbrace{Text unter der Klammer}{Text im Kontext}

\DeclareGraphicsRule{.tif}{png}{.png}{`convert #1 `dirname #1`/`basename #1 .tif`.png}

\title{Algorithmen und Programme}
\author{Protokolliert von Rouven Czerwinski}
\date{Version vom \today}                                           % Activate to display a given date or no date

\begin{document}
\maketitle
\newpage~\newpage

\tableofcontents
\listoftables
\newpage

\section{Einführung}

\begin{itemize}
 \item Kleinstcomputer (eingebettete Systeme) mit Alg. in allen Bereichen des täglichen Lebens: Taschenrechner, Handy, DvD-Player, MP3-Player, Waschmaschine, TV, Autos, Funkuhren\dots
 \item Programmierkenntnisse werden erwartet:
  \begin{itemize}
   \item[-] Programmierung und Steuerung komplexer Geräte und Maschinen
   \item[-] Erstellung interaktiver Medien (Internet, Videospiele, DVD, BluRay, E-Books\dots )
   \item[-] Verwaltung und Auswertung von Datenbanken
  \end{itemize}
\end{itemize}

\subsection{Algorithmusbegriff}
Intuitiv: Alg. = Verarbeitungsvorschrift \\
Im Alltag z.B. Kochrezept, Spielregeln, Noten, Waschmaschinenprogramme, \dots \\
Man spricht von einem Alg., wenn die Vorschrift \underline{präzise}, \underline{eindeutig}, \underline{vollständig} und \underline{ausführbar} ist. \\
~\\
\underline{Definition:}

\begin{quote}
Ein Alg. ist eine präzise formulierte Verarbeitungsvorschrift, die unter Verwendung elementarer Operationen einen Eingangszustand bzw. Einganswerte in einen Ausgangszustand bzw. Ausgangswerte überführt
\end{quote}
~\\
\underline{Formal:} Abbildung f: Eingabe $\rightarrow$ Ausgabe \\
~\\
\underline{Beispiele:}
\begin{itemize}
 \item Mathematische Formeln: $f: \mathbb{R} \times \mathbb{R} \rightarrow \mathbb{R}$ z.B. Addition zweier Zahlen $f(q,p) = q+p$
 \item Primzahlentest: $f: \mathbb{N} \rightarrow \{ ja, nein\}$ \\
  \begin{quote}
   $ f(n)=\left\{\begin{array}{cl}  \mbox{ja, falls }n \mbox{ Primzahl} \\ \mbox{nein, sonst} \end{array}\right. $
  \end{quote}
 \item  Euklidischer Alg.  ggT(x,y)
\end{itemize}
Alg. dienen zur Lösung von Problemen, sie werden als Programme so abgefasst, dass sie von einem Rechner ausgeführt werden können: \\
\begin{quote}
  $ \underbrace{\mbox{Problem} \rightarrow \mbox{Algorithmus} \rightarrow \mbox{Progr}}\mbox{amme} \rightarrow \mbox{Maschine} $
\end{quote}
Gegenstand der Vorlesung


\newpage
\section{Algorithmische Grundkonzepte}
 \subsection{Eigenschaften von Algorithmen}
  \begin{itemize}
   \item \underline{Terminiertheit} \\
    Ein Alg. terminiert, wenn er für jede Wahl von gültigen Eingabewerten nach endlich vielen Schritten anhält
   \item \underline{Determiniertheit} \\
    Ein Alg. ist determiniert, wenn er bei gleicher Eingabe stets auf das gleiche Ergebnis führt.
   \item \underline{Determinismus} \\
    Ein Alg. ist deterministisch wenn er bei gleicher Eingabe stets über die gleichen Zwischenergebnisse zum gleichen Ergebnis führt.
    \item Beispiel: Berechnung eines Terms
    \begin{itemize}
     \item[-] hält immer an $\Rightarrow$ terminiert
     \item[-] gleiches Ergebnis $\Rightarrow$ determiniert \\
      $\Rightarrow$ nicht deterministisch
    \end{itemize}
   \end{itemize}
 \subsection{Daten, Operanden und Operationen}
 Daten:
 \begin{itemize}
 \item Darstellung von Informationen im Rechner zur Eingabe, Verarbeitung und Ausgabe
 \item zB. Zahlen, Zeichen, Texte, Tabellen, Graphen, Bilder, \dots
 \item Rechnerinterne Darstellung zB. (komprimiert vs. unkomprimiert)
 \end{itemize}
 Datentyp:
 \begin{itemize}
  \item Zusammenfassung von Wertebereich und darauf def. Operationen zu einer Einheit
  \item Beispiel: Standarddatentypen: int, float, char, \dots
  \item Ein Alg. lässt sich auffassen als Anwenden von Operationenauf Objekte bestimmten Datentyps (=Operanden).
  \item Operand können Konstanten, Variablen oder Ausdrücke sein.
  \item Ausdrücke (Terme) entstehen indem Operanden mit Operationen verknüpft werden
  \item Datentypen legen die Wertemenge fest, aus der die Operanden Werte annehmen können
 \end{itemize}
\newpage
\section{Imperative Algorithmen}
\begin{itemize}
 \item Basis für imperative Programmiersprachen wie C, Pascal, Modula, Basic, PHP, \dots
 \item bekannteste/häufigste Art Alg. zu formulieren
 \item Alternative: Deklarative Programmierung z.B. \\
 funktionale Programmiersprachen wie z.B \\
 Lisp, Scheme, Haskell, \dots
\end{itemize}

\subsection{Ein- und Ausgabe}
 C++ Beispiel
 \begin{lstlisting}
int a;
cin >> a; //Eingabe des Wertes von a ueber Konsole
cout << "Sie haben" << a << "eingegeben" << ende; // Ausgabe
 \end{lstlisting}

\subsection{Verzweigungen (bedingte Anweisung)}
Mit Verzweigungen können in Abhängigkeit einer Bedingung unterschiedliche Anweisungen ausgeführt werden. \\
\underline{Syntax:} \\
\begin{lstlisting}
if (BEDINGUNG) ANWEISUNG;
else ANWEISUNG2;
\end{lstlisting}
\underline{Interpretation:} \\
Führe ANWEISUNG1 aus, falls die boolsche BEDINGUNG wahr (true) ist, ansonsten führe ANWEISUNG2 aus. (Der Else-Teil ist optional) \\
C++ Beispiel:
\begin{lstlisting}
if (a<0) cout << "a ist negativ" << endl;
else cout << "a ist positiv" << endl;

bool b = ((a % 2) == 0); // b wird true, falls a modulo 2 null ist
if(b) cout << "a ist gerade" << endl;
\end{lstlisting}

~\\
Eine ANWEISUNG darf auch ein mit Klammern \{\} zusammengefasster Block von mehreren Anweisungen sein.
\newpage
\underline{Syntax:} \\
\begin{lstlisting}
if (BEDINGUNG)
{
	Anweisung 1.1
	Anweisung 1.2
}
\end{lstlisting}

\begin{table}[h]
	\caption[ModuloTabelle]{Modulo Tabelle}
	\begin{center}
	\begin{tabular}{c|c|c|c|c}
		a & $a/2$ & a\%2 & $a/3$ & a\%3 \\
		\hline
		0 & 0 & 0 & 0 & 0\\
		1 & 0 & 1 & 0 & 1 \\
		2 & 1 & 0 & 0 & 2 \\
	\end{tabular}
	\end{center}
\end{table}

\end{document}