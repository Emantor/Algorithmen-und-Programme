\section{Einführung}

\begin{itemize}
 \item Kleinstcomputer (eingebettete Systeme) mit Alg. in allen Bereichen des täglichen Lebens: Taschenrechner, Handy, DvD-Player, MP3-Player, Waschmaschine, TV, Autos, Funkuhren\dots
 \item Programmierkenntnisse werden erwartet:
  \begin{itemize}
   \item[-] Programmierung und Steuerung komplexer Geräte und Maschinen
   \item[-] Erstellung interaktiver Medien (Internet, Videospiele, DVD, BluRay, E-Books\dots )
   \item[-] Verwaltung und Auswertung von Datenbanken
  \end{itemize}
\end{itemize}

\subsection{Algorithmusbegriff}
Intuitiv: Alg. = Verarbeitungsvorschrift \\
Im Alltag z.B. Kochrezept, Spielregeln, Noten, Waschmaschinenprogramme, \dots \\
Man spricht von einem Alg., wenn die Vorschrift \underline{präzise}, \underline{eindeutig}, \underline{vollständig} und \underline{ausführbar} ist. \\
~\\
\underline{Definition:}

\begin{quote}
Ein Alg. ist eine präzise formulierte Verarbeitungsvorschrift, die unter Verwendung elementarer Operationen einen Eingangszustand bzw. Einganswerte in einen Ausgangszustand bzw. Ausgangswerte überführt
\end{quote}
~\\
\underline{Formal:} Abbildung f: Eingabe $\rightarrow$ Ausgabe \\
~\\
\underline{Beispiele:}
\begin{itemize}
 \item Mathematische Formeln: $f: \mathbb{R} \times \mathbb{R} \rightarrow \mathbb{R}$ z.B. Addition zweier Zahlen $f(q,p) = q+p$
 \item Primzahlentest: $f: \mathbb{N} \rightarrow \{ ja, nein\}$ \\
  \begin{quote}
   $ f(n)=\left\{\begin{array}{cl}  \mbox{ja, falls }n \mbox{ Primzahl} \\ \mbox{nein, sonst} \end{array}\right. $
  \end{quote}
 \item  Euklidischer Alg.  ggT(x,y)
\end{itemize}
Alg. dienen zur Lösung von Problemen, sie werden als Programme so abgefasst, dass sie von einem Rechner ausgeführt werden können: \\
\begin{quote}
  $ \underbrace{\mbox{Problem} \rightarrow \mbox{Algorithmus} \rightarrow \mbox{Progr}}\mbox{amme} \rightarrow \mbox{Maschine} $
\end{quote}
Gegenstand der Vorlesung