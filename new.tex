\section{neue Mitschrift}
Whd: \\
statisches Array: \\
\begin{lstlisting}
int a[4]; // Größe konstant
a[0] = 35;
\end{lstlisting}

\begin{lstlisting}
int size;
cin >> size;
int b[size]; // Compilerfehler
\end{lstlisting}
stattdessen ein dynamisches Array: \\
\begin{lstlisting}
int* pa = new int[size];
pa[0] = 35;
delete[] pa; // Gibt den Sepicher wieder frei
\end{lstlisting}
Achtung! Klammern[] dürfen nicht vergessen werden; da sont nur das erste Element des Arrays freigegeben würde!
\begin{underline} In C analog zu \'new\' und \'delete\':\end{underline}
\begin{lstlisting}
int* pa = (int*)malloc(size*4); //size*4 Byte reserviert
pa[0] = 1001;
...
free(pa);
\end{lstlisting}
oder besser:
\begin{lstlisting}
int* pa = (int*)malloc(size*sizeof(int));
\end{lstlisting}
\subsection{Dynmisches Erzeugen einer Struktur}
\begin{lstlisting}
struct Person
{
    string name;
    int alter;
    ...;
}
void main()
{
    Person z; //Statisch erzeugt
    z.alter = 29;
    Person* p; //Zeiger auf eine Person
    p = new Person; //Speicher reservieren
    (*p).alter = 29; // setzt Alter auf 29 C-Syntax
    p -> alter=29; // setzt alter in C++-Syntax
    ...
    delete p;
}
\end{lstlisting}
Bisher: Liste von Daten in Arrays, Problem bei Arrays: \\
\begin{itemize}
\item[-]Die Elemente müssen alle hintereinander im Speicher abgelegt werden
\item[-]Einfügen und Löschen langsam, da die hinteren Elemente umkopiert werden müssen
\end{itemize}

\subsection{Lineare Listen} % Sollte 6.3 sein
\begin{itemize}
\item[-] Lineare Listen sind verkettete Folgen von Datenelementen
\item[-] Jedes Element enthält neben den eigentlichen Daten einen zeiger auf das nächste Element
\end{itemize}
C/C++-Beispiel
\begin{lstlisting}
struct Element
{
    string name;
    int nummer; // Personalnummer
    Element *next; // Zeiger auf nächstes Element 
}
\end{lstlisting}
Element: name nummer next \\
\begin{underline}Verkettete Liste von Elementen: \end{underline} \\
Anfang $\Rightarrow$ 1. Element | next $\Rightarrow$ ... \\
Nullzeiger (NULL bzw 0 markiert, dass ein Zeiger auf keine gültige Speicherstelle zeigt \\% 0 müsste durchgestrichen sein
\begin{underline}Erzeugen und Verketten von einer Liste mit zwei Elementen (C++):\end{underline}
\begin{lstlisting}
Element* anfang = new Element;
anfang -> name = "Peter Meier";
anfang -> nummer = 230001;
anfang -> next=NULL;
Element *zweites = new Element;
zweites -> name "Gabi Schmidt";
zweites -> nummer = 230002;
zweites -> next = NULL;
anfang -> next = zweites; // Verkettung vom ersten und zweiten Element
\end{lstlisting}
\begin{itemize}
\item[-] Größe und Verkettung können \begin{underline}während der Programlaufzeit\end{underline} verändert werden (Einfügen/Löschen von Elementen)
\item[-] Speicherplatz wird dynamisch zugeteilt
\item[-] Die einzelnen Listenelemente können sich an belibigen Stelle im Speicher befinden.
\end{itemize}
\begin{underline}Operationen auf Listen:\end{underline} \\
Füge neues Element hinter dem Element auf das Zeiger p zeigt ein. \\
C++ Beispiel
\begin{lstlisting}
void insertElemen(/Element *p, const string &name, int nummer)
{
    Element *q = new Element;
    q -> name = name;
    q -> nummer = nummer;
    q -> next = p -> next;
}
// Beispielaufruf (Einfügen eines neuen 3. elements):
insertElement(anfang->next, "Ute Müller", 230120);
\end{lstlisting}
Lösche das Element hinter dem Element auf das p zeigt:
\begin{lstlisting}
void deleteElement(Element *p)
{
    if(p -> next !NULL) // Teste ob Nachfolger existiert
    {
        Element* q = p -> next;
        p -> next = q -> next; // neu verketten
        delete q;
    }
}
\end{lstlisting}
\subsection{Bäume}
% Hier folgen irgendwann viele Bilder
\begin{itemize}
    \item[-] Ermöglicht Speicherung hierarchischer Strukturen (Gliederung in Dokumenten, Dateiverzeichnisse, verschachtelte terme, Vererbung, Organisationhierachien...)
    \item[-] Elemnt des Baumes (Knoten) sind durch gerichtete Kanten verbunden (keine Zyklen!)
    \item[-] Alle Elemnte außer der Wurzel haben einen Vorgänger
    \item[-] Binärer Baum maximal 2 Nachfolger pro Knoten
    \item[-] Definition: Ein binärer Baum B ist entweder leer oder besteht aus einer Wurzel w, sowie aus einem linken Teilbaum $B_l(w)$ und einem rechten Teilbaum $B_r(w)$.
    \item[*] \underline{Datenstrukturen}: ähnlich wie Liste jedoch 2 Nachfolger pro Element \\
C/C++ Beispiel
\begin{lstlisting}
struct Knoten
{
    int x; // zu speichernder Wer
    Knoten* links; // Zeiger auf den linken Teilbaum
    Knoten* rechts; // Zeiger auf den rechten Teilbaum
};
\end{lstlisting}
    \item[*] Eignet sich gut um sortierte Daten zu speichern und schnell Zugriffsoperationen zu ermöglichen.
    \item[*] \underline{Baumtraversierung}: Vollständiger Durchlauf eines Baumes, um sich z.B. auf alle Elemente eine Operation anzuwenden.
    \item[*] Man unterscheidet drei Traversierungsarten: preorder, inorder, postorder
    %Hier fehlen viele Bilder
    Alternative Anschauung:\\
    Preorder, inorder und postorder durchlaufen die Knoten in der gleichen Reihenfolge, unterscheiden sich jedoch darin wann die Operation auf die Knoten angewendet wird.\\
    %Hier wieder ein Bild
    \begin{lstlisting}
// Paramter *p: Zeiger auf Wurzel bzw. Teilbaum
void traverse(Knoten* p)
{
    if(p != NULL)
    {
        Operation auf p -> x; // für preorder
        traverse(p -> links);
        Operation auf p -> x; // für inorder
        traverse(p -> rechts);
        Operation auf p -> x; // für postorder
    }
}
\end{lstlisting}
\end{itemize}
\subsection{Binäre Suchbäume}
Ein binärer Baum B heißt Suchbaum wenn: \\
für alle $x \in \mbox{Be}(w) \mbox{ gilt } x < w$ und $x \in \mbox{Br}(w) \mbox{ gilt } x > w$ und Be($w$), Br($w$) Suchbäume sind. 

\underline{Einfügen neuer Elemente in B} (Struktogram)
%Hier das Struktogram einfügen
