\section{Imperative Algorithmen}
\begin{itemize}
 \item Basis für imperative Programmiersprachen wie C, Pascal, Modula, Basic, PHP, \dots \\
 \item bekannteste/häufigste Art Alge. zu formulieren
 \item Alternative: Deklarative Programmierung z.B. \\
 funktionale Programmiersprachen wie z.B \\
 Lisp, Scheme, Haskell, \dots
\end{itemize}

\subsection{Ein- und Ausgabe}
 C++ Beispiel
 \begin{lstlisting}
 	int a;
  	cin >> a; //Eingabe des Wertes von a ueber Konsole
  	cout << "Sie haben" << a << "eingegeben" << ende; // Ausgabe
 \end{lstlisting}

\subsection{Verzweigungen (bedingte Anweisung)}
Mit Verzweigungen können in Abhängigkeit einer Bedingung unterschiedliche Anweisungen ausgeführt werden. \\
\underline{Syntax:} \\
\begin{lstlisting}
 	if (BEDINGUNG) ANWEISUNG;
	else ANWEISUNG2;
\end{lstlisting}
\underline{Interpretation:} \\
Führe ANWEISUNG1 aus, falls die boolsche BEDINGUNG wahr (true) ist, ansonsten führe ANWEISUNG2 aus. (Der Else-Teil ist optional) \\
C++ Beispiel:
\begin{lstlisting}
if (a<0) cout << "a ist negativ" << endl;
else cout << "a ist positiv" << endl;

bool b = ((a % 2) == 0); // b wird true, falls a modulo 2 null ist
if(b) cout << "a ist gerade" << endl;
\end{lstlisting}
\begin{table}[h]
	\caption[ModuloTabelle]{Modulo Tabelle}
	\begin{center}
	\begin{tabular}{c|c|c|c|c}
		a & $a/2$ & a\%2 & $a/3$ & a\%3 \\
		\hline
		0 & 0 & 0 & 0 & 0\\
		1 & 0 & 1 & 0 & 1 \\
		2 & 1 & 0 & 0 & 2 \\
	\end{tabular}
	\end{center}
\end{table}
~\\
Eine ANWEISUNG darf auch ein mit Klammern \{\} zusammengefasster Block von mehreren Anweisungen sein. \\
\underline{Syntax:} \\
\begin{lstlisting}
if (BEDINGUNG)
{
	Anweisung 1.1
	Anweisung 1.2
}
\end{lstlisting}
