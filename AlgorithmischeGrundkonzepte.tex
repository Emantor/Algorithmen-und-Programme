\section{Algorithmische Grundkonzepte}
 \subsection{Eigenschaften von Algorithmen}
  \begin{itemize}
   \item \underline{Terminiertheit} \\
    Ein Alg. terminiert, wenn er für jede Wahl von gültigen Eingabewerten nach endlich vielen Schritten anhält
   \item \underline{Determiniertheit} \\
    Ein Alg. ist determiniert, wenn er bei gleicher Eingabe stets auf das gleiche Ergebnis führt.
   \item \underline{Determinismus} \\
    Ein Alg. ist deterministisch wenn er bei gleicher Eingabe stets über die gleichen Zwischenergebnisse zum gleichen Ergebnis führt.
    \item Beispiel: Berechnung eines Terms
    \begin{itemize}
     \item[-] hält immer an $\Rightarrow$ terminiert
     \item[-] gleiches Ergebnis $\Rightarrow$ determiniert \\
      $\Rightarrow$ nicht deterministisch
    \end{itemize}
   \end{itemize}
 \subsection{Daten, Operanden und Operationen}
 Daten:
 \begin{itemize}
 \item Darstellung von Informationen im Rechner zur Eingabe, Verarbeitung und Ausgabe
 \item zB. Zahlen, Zeichen, Texte, Tabellen, Graphen, Bilder, \dots
 \item Rechnerinterne Darstellung zB. (komprimiert vs. unkomprimiert)
 \end{itemize}
 Datentyp:
 \begin{itemize}
  \item Zusammenfassung von Wertebereich und darauf def. Operationen zu einer Einheit
  \item Beispiel: Standarddatentypen: int, float, char, \dots
  \item Ein Alg. lässt sich auffassen als Anwenden von Operationenauf Objekte bestimmten Datentyps (=Operanden).
  \item Operand können Konstanten, Variablen oder Ausdrücke sein.
  \item Ausdrücke (Terme) entstehen indem Operanden mit Operationen verknüpft werden
  \item Datentypen legen die Wertemenge fest, aus der die Operanden Werte annehmen können
 \end{itemize}