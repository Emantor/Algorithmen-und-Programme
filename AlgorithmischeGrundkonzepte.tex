\section{Algorithmische Grundkonzepte}
 \subsection{Eigenschaften von Algorithmen}
  \begin{itemize}
   \item \underline{Terminiertheit} \\
    Ein Alg. terminiert, wenn er für jede Wahl von gültigen Eingabewerten nach endlich vielen Schritten anhält
   \item \underline{Determiniertheit} \\
    Ein Alg. ist determiniert, wenn er bei gleicher Eingabe stets auf das gleiche Ergebnis führt.
   \item \underline{Determinismus} \\
    Ein Alg. ist deterministisch wenn er bei gleicher Eingabe stets über die gleichen Zwischenergebnisse zum gleichen Ergebnis führt.
    \item Beispiel: Berechnung eines Terms
    \begin{itemize}
     \item[-] hält immer an $\Rightarrow$ terminiert
     \item[-] gleiches Ergebnis $\Rightarrow$ determiniert \\
      $\Rightarrow$ nicht deterministisch
    \end{itemize}
   \end{itemize}
 \subsection{Daten, Operanden und Operationen}
 Daten:
 \begin{itemize}
 \item Darstellung von Informationen im Rechner zur Eingabe, Verarbeitung und Ausgabe
 \item zB. Zahlen, Zeichen, Texte, Tabellen, Graphen, Bilder, \dots
 \item Rechnerinterne Darstellung zB. (komprimiert vs. unkomprimiert)
 \end{itemize}
 Datentyp:
 \begin{itemize}
  \item Zusammenfassung von Wertebereich und darauf def. Operationen zu einer Einheit
  \item Beispiel: Standarddatentypen: int, float, char, \dots
  \item Ein Alg. lässt sich auffassen als Anwenden von Operationenauf Objekte bestimmten Datentyps (=Operanden).
  \item Operand können Konstanten, Variablen oder Ausdrücke sein.
  \item Ausdrücke (Terme) entstehen indem Operanden mit Operationen verknüpft werden
  \item Datentypen legen die Wertemenge fest, aus der die Operanden Werte annehmen können
 \end{itemize}
 ~\\
 \subsection{Standard-Datentypen}: \\
 (In den meisten Programmiersprachen vorgegeben)
 \subsubsection{Integer:} (in der Programmiersprache C/C++: int) \\
 C/C++ Beispiel: (Deklaration einer Variabln a vom Typ Integer) \\
 \begin{lstlisting}
 int a; // Kommentar
 \end{lstlisting} 
 Datentyp, Variablenname, Befehlsende \\
 Wertemenge: Z = \{\dots,-1,0,1,2,...\} (im realen Rechner nach oben und unten begrenzt) \\
 Rechenoperationen: +, -, *, /, \% (modulo), ++ (Inkrement), --(Dekrement) \\
 Zuweisungsoperator: = \\
 Vergleichsoperatorn: <, >, == (gleich), != (ungleich), <= (kleiner gleich), >=, \dots \\
 C/C++ Beispiel: \\
 \begin{lstlisting}
 int a; // Deklaration der Variablen a
 int b; // Deklaration der Variablen b
 a = 5; // Wertezuweisung: a wird auf 5 gesetzt
 b = a +8;
 \end{lstlisting}
 Variablen bestehen aus einem Namen (Referenz auf einen Speicherplatz) und einem Wert (Inhalt des Speicherplatzes)
 \begin{lstlisting}
 int c;
 c = b / a; // Division: c wird auf b/a also 2 gesetzt
 \end{lstlisting}
 Problem bei der Integer-Division: \\
 Ergebnis wird ganzzahlig abgerundet: $13 / 5$ ergibt 2, \\
 Allg. zur Berechnung des Restes der ganzzahligen Division b/a:
 \begin{lstlisting}
 int rest;
 rest = b - (b/a)*a;
 \end{lstlisting}
 Der Rest kann in C/C++ auch durch den Modulo -Operator \% beschrieben werden
 \begin{lstlisting}
 rest = b % a;
 \end{lstlisting}
 
 \subsubsection{Real:} (in C/C++: float und double) \\
 Wertemenge: Q (I'm realen Rechner nur eine Teilmenge von Q) \\
 Operatoren wie oben \\
 C/C++ Beispiel
 \begin{lstlisting}
 double c,d; // Deklaration der Var. c und d
 c = 0,3; // c wird auf 0,3 gesetzt
 d = (4,5 - c)*1,5; // d <- 6,3 
 \end{lstlisting}
 
 \subsubsection{Character:} (in C/C++: char) \\
 Wertemenge zB. \{'a', 'b',\dots,'A','B', \dots,'1','2',\dots,'\#', \dots\}
 \begin{lstlisting}
 char e;
 e = 'z';
 char f = '#'
 \end{lstlisting}
 
 \subsubsection{String:} (in C kein Standarddatentyp, statt dessen Array vom Typ char, in C++ std::string) \\
 Wertemenge: Zeichenketten, zB. ''Hallo'', ''Guten Tag'', \dots \\
 C++ Beispiel:
 \begin{lstlisting}
 char wort[6] = "Hallo"; //Deklaration als char-array
 std::string s="Guten Tag"; // Deklaration als std::string
 \end{lstlisting}
 
 \subsubsection{Boolean:} (in C++ bool; in C kein Standarddatentyp, stattdessen int) \\
 Wertemenge: \{true, false\} btw. \{0,1\} \\
 Einstelliger: ! (not)
 Zweistellige Verknüpfungensoperatoren: \&\& (and), || (or), == (gleich), != (ungleich), \dots \\
 C++ Beispiel
 \begin{lstlisting}
 bool ergebnis,op1,op2;
 op1 = true;
 op2 = !op1; // op2 wird false
 ergebnis = op1 && op2; // ergebnis wird false
 Ergebnis = op1 || op2; // ergebnis wird true
 \end{lstlisting}
 Verknüpfungstabellen: \\
\begin{table}[p]
	\caption[And Tabelle]{And Tabelle}
	\begin{center}
	\begin{tabular}{cc|c}
		op1 & \&\& op2 & Ergebnis	\\
		\hline
		true & true & true\\
		true & false & false	\\
		false & true & false \\
		false & false & false \\
	\end{tabular}
	\end{center}
\end{table}
\begin{table}[p]
	\caption[Or Tabelle]{Or Tabelle}
	\begin{center}
	\begin{tabular}{cc|c}
		op1 & || op2 & Ergebnis	\\
		\hline
		true & true & true\\
		true & false & true	\\
		false & true & true \\
		false & false & false \\
	\end{tabular}
	\end{center}
\end{table}
\begin{table}[p]
	\caption[Negation Tabelle]{Negation Tabelle}
	\begin{center}
	\begin{tabular}{c|c}
		!op1 &  Ergebnis	\\
		\hline
		true & false\\
		false & true	\\
	\end{tabular}
	\end{center}
\end{table}
\newpage
Die Reihenfolge der Auswertung von Ausdrücken ergibt sich durch Klammerung und Vorrangregeln (Punkt vor Strich usw z.B. $(a*b +c*d/e)$) \\
Beispiel: \\
\begin{lstlisting}
bool Ergebnis,op1,op2,op3;
Ergebnis = op1 || op2 && op3;
Ergebnis = op1 || (op2 && op3);
\end{lstlisting}
 