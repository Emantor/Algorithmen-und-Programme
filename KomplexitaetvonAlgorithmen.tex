\section{Komplexität von Algorithmen}
\subsection{Einführung}
Alg1:
\begin{lstlisting}
void Alg1(int n)
{
	for(int i=0; i < n*4; i=i+1)
	{
		tue_etwas(...);
	}
}
\end{lstlisting}
Alg2:
\begin{lstlisting}
void Alg2(int n)
{
	for(int i=0; i < n; i=i+1)
	{
		for(int j=0; j<n; j=j+)
		{
			tue_etwas(...);
		}
	}
}
\end{lstlisting}
Aufwandsvergleich(Anzahl der Aufrufe von \texttt{tue\_etwas}): \\
\begin{table}[h]
	\caption[Aufwandsvergleich der Algorithmen]{Aufwandsvergleich}
	\begin{center}
	\begin{tabular}{c|ccccccc}
		n & 1 & 2 & 3 & 4 & 5 & 6 & 7\\
		\hline
		Alg1 & 4 & 8 & 12 & 16 & 20 & 24 & 28\\
		Alg2 & 1 & 4 & 9 & 16 & 25 & 36 & 49\\
	\end{tabular}
	\end{center}
\end{table}
~\\
Alg1 $\longleftarrow$ linear (4n) \\
Alg2 $\longleftarrow$ quadratisch ($n^2$) \\
\begin{itemize}
\item Aufwand abhängig von
	\begin{itemize}
	\item[-] Größe n der Eingabe, Anzahl der Daten usw.
	\item[-] der Komplexität des Algorithmus
	\end{itemize}
\item Unterscheidung zwischen 
	\begin{itemize}
	\item[(a)] Zeitkomplexität (Laufzeit der Alg. s.o.)
	\item[(b)] Speicherkomplexität (benötigte Speichermenge)
	\end{itemize}
\item 90/10-Regel beim Programmieren: \\ 
\glqq Ungefähr 90\% der Laufzeit wird in ca. 10\% des Programmcodes verbraucht\grqq
\end{itemize}
