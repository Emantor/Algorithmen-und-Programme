\section{Abstrakte Datentypen und objektorientierte Programmierung}
\subsection{Abstrakte Datentypen (ADT)}
\begin{itemize}
    \item Ein primitiver Datentyp (z.B. int, float, string) besteht aus Wertebereich + Menge von Operationen.
    \item Ein ADT ist die Übertragung dieses Konzepts auf (eigene) komplexe Datentypen (z.B. Datensätze, Vektoren, Bilder) 
    \item ADT fasst Datenstrukturen und zugehörige Operationen zu einer Einheit zusammen
    \item Die Menge der Operationen wird auch Schnittstelle (Interface) genannt.
\end{itemize}
\underline{Hauptvorteil eines ADTs}
\begin{itemize}
    \item \underline{Kapselung}: Daten werden nur über definierte Schnittstellen bearbeitet. \\
    $\Rightarrow$ Benutzer weiß was ADT tut, muss aber nicht wissen wie (black box)
    \item \underline{Schutz}: Die Gefahr, Daten ungewollt zu verändern oder Programmierfehler zu begehen nimmt ab.
    \item \underline{Modularität}: Übersichtlicher, geringe Vernetzung zwischen Daten und Programm $\Rightarrow$ Interne Repräsentation der Daten (z.B. Array oder Liste) kann leicht geändert werden.
    \item \underline{Wiederverwertbarkeit}
\end{itemize}
